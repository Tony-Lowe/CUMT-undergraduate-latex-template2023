\documentclass[preprint,MD,numbers,times,AutoFakeBold]{cumtthesis}
\usepackage{fontspec}
\usepackage{textcomp, gensymb}

\setmainfont{Times New Roman}
\usepackage{listings}
\usepackage{multirow}
\usepackage{indentfirst}
\usepackage{tikz}
\usepackage{etoolbox}
\usepackage{color}
% 使用更新的伪代码包写法
\usepackage{algpseudocodex}
\usepackage{float}
\usepackage{rotating}
\usepackage{booktabs}
\usepackage{enumerate}
\usepackage[a4paper,left=3.17cm,right=3.17cm,top=2.54cm,bottom=2.54cm]{geometry}
\usetikzlibrary{matrix,calc,shapes,backgrounds,patterns,positioning,decorations.pathreplacing}
%=================================== 数学符号 =================================%
\newcommand{\rtn}{\mathrm{\mathbf{R}}}
\newcommand{\N}{\mathrm{\mathbf{N}}}
\newcommand{\AS}{~\mathrm{a.s.}}
\newcommand*{\PR}{\mathrm{\mathbf{P}}}
\newcommand*{\EX}{\mathrm{\mathbf{E}}}
\newcommand*{\dif}{\,\mathrm{d}}
\newcommand*{\F}{\mathcal{F}}
\newcommand*{\prs}{\dif\PR-\mathrm{a.s.}}
\newcommand*{\pts}{\dif\PR\times\dif t-\mathrm{a.e.}}
\newcommand{\Ito}{It\^{o}}
\newcommand{\tT}[1][0]{[#1,T]}
\newcommand{\intT}[2][T]{\int^{#1}_{#2}}
\newcommand{\s}{\mathcal{S}}
\newcommand{\me}{\mathrm{e}}
\newcommand{\one}[1]{{\bf 1}_{#1}}
\newcommand{\Mm}{{\rm M}}
\newcommand{\circled}[2][]{\tikz[baseline=(char.base)]
    {\node[shape = circle, draw, inner sep = 1pt]
    (char) {\phantom{\ifblank{#1}{#2}{#1}}};%
    \node at (char.center) {\makebox[0pt][c]{#2}};}}
\robustify{\circled}
\setlength{\intextsep}{10.0pt plus 2.0pt minus 2.0pt}


\let\oldcite=\cite
\renewcommand\cite[1]{
  \textsuperscript{\oldcite{#1}}
}
\DeclareMathOperator*{\sgn}{sgn}
%=================================== 数学符号 =================================%
%=================================== Code Style ==============================%
\definecolor{mygreen}{rgb}{0,0.6,0}
\definecolor{mygray}{rgb}{0.5,0.5,0.5}
\definecolor{mymauve}{rgb}{0.58,0,0.82}
\lstset{
  aboveskip=3mm,
  belowskip=3mm,
  showstringspaces=false,
  columns=flexible,
  basicstyle={\normalsize\ttfamily},
  numbers=left,
  frame=L,
  numberstyle=\tiny\color{mygray},
  keywordstyle=\color{blue},
  commentstyle=\color{mygreen},
  stringstyle=\color{mymauve},
  breaklines=true,
  breakatwhitespace=true,
  tabsize=3,
  framextopmargin=2pt,framexbottommargin=2pt,abovecaptionskip=-3pt,belowcaptionskip=3pt,  
  xleftmargin=1em, % 设定listing左右的空白
}

\begin{document}
\input{tex/frontinfo}

\input{tex/frontmatter}



%--------插入中文目录
\tableofcontents
%--------插入英文目录
\tableofecontents

% --------插入图清单
\listoffigures
% --------插入表清单
\listoftables

% --------插入变量注释表
\begin{notation}[2.5cm]
\item[$min\_s$] 最小支持度
\item[$min\_c$] 最小置信度
\item[$R$] 空间邻近关系
\item[$row\_instance$]  行实例
\item[$table\_instance$] 表实例
\item[$PR(c,f_i)$] 空间参与率
\item[$PI(c)$] 空间参与度
\item[$min\_prev$] 最小参与度
\item[$\delta_i$] 流入量
\item[$\omega_i$] 流出量
\item[$\psi_i$] 流入流出比
\item[$Q$] 模块值
\item[$\Delta Q$] 模块增益值
\end{notation}

%================================== 正文开始 ==================================%

\mainmatter
% 写新章节后必须在这里include tex文件,否则无法编译
\include{tex/chap1}
\include{tex/chap2}
\include{tex/chap3}
\include{tex/chap4}

\backmatter

% 参考文献格式使用cumt.bst
\bibliographystyle{biblio/cumt}
\bibliography{biblio/RefExam}

\pagestyle{cumt@trans}

\begin{translation}

\section*{英文原文}

\par\vskip1.5em
\parindent2em

\parindent=0em \parbox[t]{\textwidth}{\bfseries\heiti\zihao{-2}\centering Learning Progressive Point Embeddings for 3D Point Cloud Generation}

  \par\vskip1.5em
  \parindent2em


\parindent=0em \parbox[t]{\textwidth}{\heiti\zihao{5}\centering Cheng Wen, Baosheng Yu, Dacheng Tao}

  \par\vskip1.5em
  \parindent2em
  
\section*{Abstract}

Generative models for 3D point clouds are extremely important for scene/object reconstruction applications in autonomous driving and robotics. Despite recent success of deep learning-based representation learning, it remains a great challenge for deep neural networks to synthesize or reconstruct high-fidelity point clouds, because of the difficulties in 1) learning effective pointwise representations; and 2) generating realistic point clouds from complex distributions. In this paper, we devise a dual-generators framework for point cloud generation, which generalizes vanilla generative adversarial learning framework in a progressive manner. Specifically, the first generator aims to learn effective point embeddings in a breadth-first manner, while the second generator is used to refine the generated point cloud based on a depth-first point embedding to generate a robust and uniform point cloud. The proposed dual-generators framework thus is able to progressively learn effective point embeddings for accurate point cloud generation. Experimental results on a variety of object categories from the most popular point cloud generation dataset, ShapeNet, demonstrate the state-of-the-art performance of the proposed method for accurate point cloud generation.

\subsection*{3.1 Discriminator}

In adversarial learning, the discriminator distinguishes whether the input (e.g., a point cloud) is produced by the generator or sampled from the ground truth, while the generator requires to fool the discriminator by generating a high-fidelity point cloud. Recently, [1] has investigated GAN-based architectures for point cloud generation: the discriminator first embeds the input points into high dimension in a pointwise manner, and then aggregates the feature vectors of previous step to obtain the global features, e.g., using the symmetric operation [29]. However, the above architecture usually fails to capture the local information implied in geometry structures. To overcome this, we simultaneously extract global and local point features to learn effective point representations.

As shown in Fig.2, to extract effective feature representations from point clouds, we utilize multiple MLP blocks with residual connections to learn both low- and high-level features in a pointwise manner. The input of each pointwise MLP block is a set of point features with the size$ N \times C_{in}$ and the output size $N \times C_{out}$, where $C_{in}$ and$C_{out}$ denote the numbers of channels in the input and output feature attributes, respectively. After each MLP block, we also attach a point convolution layer or the set abstraction (SA) layer [30] to extract local features by aggregating the information of neighbor points. Specifically, each point convolution layer consists of a sampling operation and a grouping operation, as shown in Fig. 3. The input size of this layer is $N \times C_{in}^′$ and the output size is $N_s \times C_{out}^′$, where $N_s$ indicates the number of sampled points.

\begin{equation}
    L_{dis}=L_s(p,\hat{p})+L_p(\mathbb{E}[p_i],\mathbb{E}[\hat{p_i}])\tag{1}
\end{equation}

\newpage

\section*{中文译文}

\par\vskip1.5em
\parindent2em

\parindent=0em \parbox[t]{\textwidth}{\bfseries\heiti\zihao{-2}\centering 面向三维点云生成的渐进式点嵌入学习}

  \par\vskip1.5em
  \parindent2em


\parindent=0em \parbox[t]{\textwidth}{\heiti\zihao{5}\centering Cheng Wen, Baosheng Yu, Dacheng Tao}

  \par\vskip1.5em
  \parindent2em

\section*{摘要}

三维点云的生成模型对于自动驾驶和机器人领域的场
景/物体重建应用极其重要。尽管最近基于深度学习的表
示 学 习 取 得 了 成 功 , 但 对 于 深 度 神 经 网 络 来 说 , 合 成 或
重建高保真的点云仍然是一个巨大的挑战,因为 1)学 习
有 效的 点 态 表 示 ;以 及 2)从 复 杂 的 分 布中 生 成 逼 真 的点 云 。
本 文 设 计 了 一 个 用 于 点 云 生 成 的 双 生 成 器 框 架 , 以 渐 进
式 的方 式 泛 化 了 vanilla 生 成 对 抗 学 习 框架 。 具 体 来说 ,
第一个生成器旨在以广度优先的方式学习有效的点嵌入,
而 第 二 个 生 成 器 用 于 基 于 深 度 优 先 的 点 嵌 入 来 细 化 生 成
的 点 云 , 以 生 成 鲁 棒 和 均 匀 的 点 云 。 因 此 , 所 提 出 的 双
生 成 器 框 架 能 够 逐 步 学 习 有 效 的 点 嵌 入 , 以 准 确 地 生 成
点 云。 在 最 流行 的 点 云 生成 数 据 集 ShapeNet 的 各 种对 象
类 别 上 的 实 验 结 果 表 明 , 所 提 出 的 方 法 在 精 确 点 云 生 成
方面具有最先进 的性 能 。
%\include{tex/resume}

\GuanJianCi{浮选,旋流,分选机理,浮选动力学,矿物分选}
\BingLieTiMing{Analysis and Data Mining about Spatial Data of Sina Weibo Based on Spatial-Spark}
\LunWenYuZhong{中文}
\XueHao{TS14160005}
\PeiYangDanWeiDaiMa{10290}
\PeiYangDanWeiDiZhi{江苏省徐州市}
\XueZhi{四年}
\LunWenTiJiaoRiQi{2022 年 6 月}
\DaBianWeiYuanHuiChengYuan{王永波,杨化超,陈国良,刘志平}
\LunWenZiZhu{无}
%\XueWeiShouYuDanWeiMingCheng{}
%\XueWeiShouYuDanWeiDaiMa{}
%\XueWeiJiBie{}
%\LunWenTiMing{}
%\PeiYangDanWeiMingCheng{}
%\YouBian{}
%\XueWeiShouYuNian{}

% \makebackcover
%\printindex
\clearpage
\end{document}
